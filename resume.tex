\documentclass[11pt]{res} % default is 10 pt
%\usepackage{helvetica} % uses helvetica postscript font (download helvetica.sty)
%\usepackage{newcent}   % uses new century schoolbook postscript font 
\usepackage{geometry}
\usepackage{endnotes}
\setlength{\textheight}{9.5in} % increase text height to fit resume on 1 page
\newsectionwidth{0pt}  % So the text is not indented under section headings
{\newgeometry{left=0.25in,right=1.3in,top=0.5in,bottom=0.5in}

\let\footnote=\endnote

\newcommand{\footnoteremember}[2]{
\footnote{#2}
\newcounter{#1}
\setcounter{#1}{\value{endnote}}
}
\newcommand{\footnoterecall}[1]{
\footnotemark[\value{#1}]
}

\begin{document} 
 
\name{SEBASTIEN DROUYER}
\address{5 rue du boulevard \\  69100, Villeurbanne\\ France \\ (+33) 6 80 90 78 17}

                                             
\begin{resume}
 
\section{EDUCATION} 
 \noindent INSA LYON (IF) - Lyon, France \\
Institut National des Sciences Appliqu\'ees de Lyon / National Institute of Applied Sciences of Lyon\\
Master's Degree in Computer Science, September 2008 - June 2011 \\
School ranked 3 / 126 in French engineering schools according to the Usine Nouvelle newspaper\footnoteremember{usine_nouvelle}{Usine Nouvelle's online ranking: http://www.usinenouvelle.com/comparatif-des-ecoles-d-ingenieurs-2013 (french)}

 \noindent INSA ROUEN (SIB) - Rouen, France \\
Institut National des Sciences Appliqu\'ees de Rouen / National Institute of Applied Sciences of Rouen\\
Preparatory school, September 2006 - June 2008 \\
International section (half the courses were taught in English) \\
School ranked 53 / 126 in French engineering schools according to the Usine Nouvelle newspaper\footnoterecall{usine_nouvelle}

 \noindent LFKL - Kuala Lumpur, Malaysia \\
Lyc\'ee Fran\c{c}ais de Kuala Lumpur / French School of Kuala Lumpur (High school)\\
Core curriculum in science and technology with honours, September 2004 - June 2006

\section{EXPERIENCE} 
 RESEARCH AND DEVELOPMENT ENGINEER - Novius, Lyon, France \\
December 2010 - Present
\vspace{0.05in}
   \begin{itemize} \itemsep -2pt  % reduce space between items
   \item Conception and implementation of an Enterprise social software, ResoNova\footnote{ResoNova's website: http://www.resonova.fr (french)}.
Used internally as well as by various customers such as banks (Banque postale) or
Research Institutes (Bioaster). I also trained two developers to enable them
to configure and extend the software. Technical support.
   \item Conception and implementation of an open source Content Management System named Novius OS\footnote{Novius OS' website: http://www.novius-os.org/ (english) \\
Novius OS' youtube channel (presentation videos): https://www.youtube.com/user/NoviusOS (english)}.
Part of the core team, composed by 3 engineers and the Chief Information Officer. The software was used by the company
developers to implement new websites. I also trained 12 developers and took part in the redaction
of the online documentation\footnote{Novius OS documentation website: http://docs.novius-os.org/ (english)} as well as in the open-source community management.
   \item FuelPHP (open source PHP framework) contributor, trainer and speaker. Trained internal development teams.
Conference speaker (see presentations section).
 \end{itemize}

RUBY ON RAILS DEVELOPER - Safecoms, Bangkok, Thailand \\ 
June 2010 - October 2010
\vspace{0.05in}
 \begin{itemize} \itemsep -2pt
   \item Implemented new functionalities and improved performances to a Customer Relationship Management system named Peppercan\footnote{Peppercan's website: http://www.peppercan.com/ (english)}.
  \item  Consultant at Venda, reviewed tickets in order to improve performance and customer satisfaction.
\end{itemize} 

UNITY DEVELOPER - Lumai Prod, Bangkok, Thailand \\ 
June 2009 - August 2009
\vspace{0.05in}
 \begin{itemize} \itemsep -2pt
   \item Implemented new functionalities and improved performances to a Wii video game.
\end{itemize} 

\section{PRESENTATIONS} 

FUELPHP: A FRAMEWORK, YES - AUTOMAGIC, NO! \\
French conference introducing the FuelPHP framework\footnote{FuelPHP conference' slides: http://www.slideshare.net/novius-os/rmll-2012-confrence-fuelphp (french)}
\vspace{0.05in}
 \begin{itemize} \itemsep -2pt
  \item November 2012: PHPTour event in Nantes\footnote{FuelPHP conference's audio recording: http://youtu.be/TbIFqYwz-SI (french)}
  \item July 2012: Libre Software Meeting event in Geneva\footnote{FuelPHP conference's video recording: http://youtu.be/BbmZiJpTGSY (french)}.
\end{itemize} 
 

\section{CONTESTS AND AWARDS} 
 
INTEL PERCEPTUAL COMPUTING CHALLENGE: FINALIST + STILL ONGOING \\
June 2013 - September 2013 \\
International software creation contest  (750 finalists\footnoteremember{intel_perceptual}{Intel Perceptual computing challenge's website: https://perceptualchallenge.intel.com/ (english)}). \\
Conception and implementation of a software maximizing the Intel Perceptual Computing SDK\footnoterecall{intel_perceptual}
(gesture tracking, face tracking, voice recognition) : my entry was Perceptual Earth Facts \footnote{Perceptual Earth Facts' presentation video: http://youtu.be/k2dnIkJhJyk (english) \\
Perceptual Earth Facts' text presentation: http://drouyer.com/perceptual-challenge/perceptual-earth-facts/help.jpg (english)}. \\
{\bf Winnings: Creative* Interactive Gesture Camera (worth \$149) for now + still ongoing}

NASA TOURNAMENT LAB - ROBONAUT CHALLENGE: 2ND AND 1ST PLACE\footnoteremember{robonaut_rankings}{Original contest rankings: http://community.topcoder.com/longcontest/stats/?module=ViewOverview\&rd=15652 \\
Original contest registrants: \\ http://community.topcoder.com/longcontest/'module=ViewProblemStatement?module=ViewRegistrants\&rd=15652 \\
Extended contest rankings: http://community.topcoder.com/longcontest/stats/?module=ViewOverview\&rd=15611 \\
Extended contest registrants:\\ http://community.topcoder.com/longcontest/'module=ViewProblemStatement?module=ViewRegistrants\&rd=15611} \\
May 2013 \\
International TopCoder marathon challenge (1190 registrants, 40 competitors\footnoterecall{robonaut_rankings}\footnoteremember{topcoder_comp_reg}{In TopCoder competitions: Registrants are people who subscribed to the contest, but did not necessarily submitted anything. Competitors are people whose submission was accepted (sent on time, compilation successful, respected limitations...).}). \\
2nd on original contest, 1st on the extended contest. \\
Conception and implementation of a vision algorithm capable of detecting leds and buttons on a taskboard.
The algorithm is intended for Robonaut 2, an humanoid robot NASA wants to send to the ISS in order to assist
astronauts\footnote{Contest's presentation video: http://youtu.be/uXZlyyJohAI \\
Contest's problem statement: http://www.topcoder.com/iss/robonaut/} \\
{\bf Winnings: \$3900}

NASA TOURNAMENT LAB - LONGERON CHALLENGE: 17TH PLACE\footnoteremember{longeron_rankings}{Contest rankings: http://community.topcoder.com/longcontest/stats/?module=ViewOverview\&rd=15520 \\
Contest registrants:\\
http://community.topcoder.com/longcontest/'module=ViewProblemStatement?module=ViewRegistrants\&rd=15520} \\
January 2013 - February 2013 \\
International TopCoder marathon challenge (4056 registrants, 459 competitors\footnoterecall{topcoder_comp_reg}\footnoterecall{longeron_rankings}). \\
Conception and implementation of an algorithm optimizing the ISS solar collectors positions in order to
maximize the generated power\footnote{Contest's presentation video: http://youtu.be/qiFDrwnUgUc (english) \\
Contest's problem statement: http://www.topcoder.com/iss/longeron/ (english)}

GITHUB GAME OFF: 1ST PLACE\footnoteremember{github_rankings}{Contest's winners: http://github.com/blog/1337-github-game-off-winners (english)} \\
November 2012 \\
International game creation contest (1300 registrants, 200 competitors \footnoterecall{github_rankings}). \\
Conception, design and implementation of a web unity game\footnote{Contest's description: https://github.com/blog/1303-github-game-off (english) \\
My contest entry: http://sebastien.drouyer.com/hotfix/ (english)} \\
{\bf Winnings: Mac Mini (worth \$700)}

NASA TOURNAMENT LAB - USPTO ALGORITHM CHALLENGE: 6TH PLACE\footnoteremember{uspto_rankings}{Contest rankings: http://community.topcoder.com/longcontest/stats/?module=ViewOverview\&rd=15087 \\
Contest registrants: \\
http://community.topcoder.com/longcontest/'module=ViewProblemStatement?module=ViewRegistrants\&rd=15087} \\
March 2012 - April 2012 \\
International TopCoder marathon challenge (150 registered, 28 competitors\footnoterecall{topcoder_comp_reg}\footnoterecall{uspto_rankings}). \\
Conception and implementation of a vision algorithm capable of detecting figures and their labels on a patent paper\footnote{Problem statement: http://community.topcoder.com/longcontest/?module=ViewProblemStatement\&rd=15087\&pm=11839 (english)}

INFOTEL MOBILE APPLICATION CONTEST: 2ND PLACE\footnoteremember{infotel_mobile}{Contest webpage (subject + rankings): http://www.infotel.com/rejoignez-nous/concours-etudiants/concours-2010-2011/ (french)} \\
February 2011 - May 2011 \\
National Student contest (15 competitors teams). \\
Conception, design and implementation of a mobile application allowing the users to buy eco-responsible food\footnoterecall{infotel_mobile} \\
{\bf Winnings: Galaxy Tab (worth \$400)}

INFOTEL WEB APPLICATION CONTEST: 1ST PLACE\footnoteremember{infotel_web}{Contest webpage (subject + rankings): http://www.infotel.com/rejoignez-nous/concours-etudiants/concours-2009-2010/ (french)} \\
January 2010 - May 2010 \\
National Student contest (15 competitors teams). \\
Conception, design and implementation of a web application allowing to store and manage documents. I used
the open-source software Tesseract (OCR recognition) in order to transform scanned forms (pictures) into usable data\footnoterecall{infotel_web} \\ 
{\bf Winnings: Toshiba Tecra computer (worth \$1000)}

\newpage
\section{MASSIVE OPEN ONLINE COURSES} 
 
COURSERA - UNIVERSITY OF WASHINGTON: COMPUTATIONAL NEUROSCIENCE\footnote{Course website: https://www.coursera.org/course/compneuro (english)} \\
Mai 2013 - June 2013: Statement of Accomplishment (97\%)

COURSERA - UNIVERSITY OF EDINBURGH: ARTIFICIAL INTELLIGENCE PLANNING\footnote{Course website: https://www.coursera.org/course/aiplan (english)} \\
February 2013 - March 2013: Statement of Accomplishment (97.6\%, with distinction,  98th percentile of passing students)

COURSERA - STANFORD: ALGORITHM, DESIGN AND ANALYSIS, PART 1\footnote{Course website: https://www.coursera.org/course/aiplan (english)} \\
February 2013 - March 2013: Statement of Accomplishment (97\%)

COURSERA - BERKELEY: SAAS \\
February 2012 - April 2012: Statement of Accomplishment (73\%)

COURSERA - STANFORD: MACHINE LEARNING\footnote{Course website: https://www.coursera.org/course/ml (english)} \\
October 2011 - January 2012: Statement of Accomplishment (100\%)
 
\section{HOBBIES} 
 
Bodybuilding, bicycle, travels, open-source contributor
 

\end{resume}

%\def\notesname{End notes}
\theendnotes

\end{document}











